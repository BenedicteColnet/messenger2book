\section*{Mardi 7 juillet\markboth{\MakeLowercase{Mardi 7 juillet}}{}}
  \vspace*{0.2cm} \begin{leftbubbles} Du coup si on résume : \\* \emoji[ios]{1F682} Départ J-0 à 20h50 de Paris Austerlitz par le train de nuit\\* \emoji[ios]{1F682} Arrivée J-1 à 8h22 à Briançon\\* Étape J-1 :\\    -  \texttt{LINK}\\    - Marche d'approche/entrée dans le Queyras : ça monte ça monte ça monte, puis on redescend dans la vallée ou on fait dodo\\    - Peut être raccourcie de 2-3 km et 500m de D- en remplaçant camping par bivouac\\* Étape J-2 :\\    -  \texttt{LINK}\\    - Quasi boucle au dessus de la vallée d'Arvieux, on monte voir l'alpage et on redescend vers le bivouac dans la vallée\\    - Bivouac pas loin d'un village\\    - Étape plutôt facile,  surtout si on raccourci le J-1\\* Étape J-3 :\\    -  \texttt{LINK}\\    - On monte voir un lac d'altitude, puis on descend dormir à côté de Chateau Queyras.\\    - Probablement la plus grosse étape (couplée au courbatures des 2 premiers jours ça va être rigolo \emoji[ios]{1F62C})\\    - Camping à Chateau Queyras\\* Étape J-4 :\\    -  \texttt{LINK}\\    - Ça monte/ça descend/ça remonte. Bivouac dans un alpage d'altitude.\\    - Plus gros D+ du parcours, coupé en 2 (+500 puis +1000)\\* Étape J-5 :\\    -  \texttt{LINK}\\    - On a beaucoup monté la veille, on redescend tout pour arriver à la gare\\    - Arrivée Gare de Montdauphin-Guillestre\\* \emoji[ios]{1F682} Départ J-5 à 20h36 de la gare de Montdauphin-Guillestre\\* \emoji[ios]{1F682} Arrivée J-6 à 6h58 à Paris Austerlitz. \hspace{0.5cm}\hfill{\textcolor{mygray}{{\footnotesize 15:54}}} \end{leftbubbles}\vspace*{2cm}  