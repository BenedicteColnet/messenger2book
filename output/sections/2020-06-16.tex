\section*{Mardi 16 juin}

\begin{leftbubbles}
La pandémie du covid-19 est une maladie nouvelle pour laquelle la recherche de traitement a remis au coeur du débat - médiatique notamment - le processus d’évaluation d’un médicament. Alors que l’un des seuls moyens de sauver des patients semble être de placer les personnes gravement atteintes en salle de réanimation et sous respirateurs, la recherche de traitement a suscité un débat mondial avec plusieurs molécules au centre du jeu, en particulier l’hydroxychloroquine. Réutiliser une molécule déjà sur le marché, mais pour traiter une autre maladie s’appelle le repurposing, ce qui est donc le cas des nombreuses molécules proposées pour traiter la covid-19. Lors de cette pandémie l’opinion publique et la sphère politique se sont approprié la question de “Faut-il ou non utiliser l'hydroxychloroquine ?”. Pourtant cette question est usuellement celle de l’expert puisqu’il existe des règles méthodologiques et éthiques préalables à l’usage d’un médicament ou d’un traitement. Finalement la question fondamentale dans ce débat serait de démêler si cette la polémique résulte uniquement d’un moment de panique général, ou bien si il y avait là la manifestation d’une crise plus profonde, qui serait celle des méthodes d’homologation de nouveaux traitements. \ \ Dans tous les cas, il est certain que ce débat a permis de mettre sur le devant de la scène des enjeux intrinsèques à l’évaluation des médicaments. Par exemple la lourdeur et la lenteur de la procédure d’essais cliniques, qui semblait inconnu du grand public. Ou encore le fait que les essais cliniques en eux-mêmes sont complexes et multifactoriels : va-t-on considérer que si la charge virale du patient diminue, alors le médicament fonctionne ? Ou souhaite-t-on au contraire regarder si des vies sont sauvées ? Enfin, ce débat scientifique mondial aura également laissé une place forte à des études dont les méthodologies sont nouvelles, comme le montre l’article dans The Lancet où la méthodologie a été vivement critiquée par la suite entraînant une rétractation de l’article [source XXX].\ \ Les méthodologies actuelles d’évaluation du médicament sont donc encore discutées, ce qui ne paraît pas tant surprenant au regard de leur origine relativement récente. Et si le socle méthodologique qui fonde la réponse à la question “Qu’est ce qui me permet de dire qu’un traitement fonctionne ?” peut parfois vaciller, ceci est à mettre au regard d’évolution de nature des médicaments. Depuis quelques dizaines d’années deux mutations majeures affectent la nature des nouveaux médicaments qui arrivent sur le marché. D’une part les médicaments qui arrivent ne sont plus des molécules, mais des composants biologiques, comme par exemple des enzymes, des protéines, ou bien des anticorps. Ces thérapies peuvent parfois remettre en cause les méthodologies classiques d’évaluation du médicament. D’autre part, les médicaments qui arrivent sur le marché touchent de plus en plus de maladies dites orphelines selon la Food and Drug Administration, c’est-à-dire qu’un médicament arrivant sur le marché va toucher un nombre plus restreint de malades que précédemment. Si certains y voient un meilleur traitement parce que plus spécifique, d’autres appellent ce phénomène la fin du blockbuster, c’est-à-dire la fin de médicament à usage vaste. Ces changements ont en fait des impacts sur toute la chaîne de développement d’un médicament, notamment économique : si le temps de recherche et de développement est le même, comment rentabiliser ces efforts sans voir une inflation du prix des traitements ? Et si les traitements touchent des maladies restreintes, comment effectuer de vastes essais cliniques ?\\
\end{leftbubbles} \begin{rightbubbles}
Lois mathématiques --\\
\end{rightbubbles}